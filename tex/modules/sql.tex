\section{SQL}

% \vspace{\baselineskip}
Creating Tables:
\begin{easylist}[itemize]
\ListProperties(Style*=$\bullet$ , FinalMark={)})
& \texttt{CREATE TABLE customers (id INTEGER PRIMARY KEY \textit{AUTOINCREMENT},\newline name TEXT, age INTEGER, weight REAL);}
\end{easylist}

% \vspace{\baselineskip}
Inserting Data:
\begin{easylist}[itemize]
\ListProperties(Style*=$\bullet$ , FinalMark={)})
& \texttt{INSERT INTO customers VALUES (73, "Brian", 33, 67.5);}
& \texttt{INSERT INTO customers (name, weight) VALUES ("Brian", 67.5);}
\end{easylist}

\vspace{\baselineskip}
Querying Data:
\begin{easylist}[itemize]
\ListProperties(Style*=$\bullet$ , FinalMark={)})
& \texttt{SELECT * FROM customers;}
& \texttt{SELECT * FROM customers LIMIT 5;}
& \texttt{SELECT age, name FROM customers;}
& \texttt{SELECT name, age + weight AS "calculation" FROM customers;}
\newline
& \texttt{SELECT * FROM customers WHERE age > 21;}
& \texttt{SELECT * FROM customers WHERE age > 21 AND age < 30;}
& \texttt{SELECT * FROM customers WHERE age > 21 OR name="Brian";}
& \texttt{SELECT * FROM customers WHERE NOT name="Brian";}
\newline
& \texttt{SELECT * FROM customers WHERE name IN ("Bill", "Bob", "Brian");}
& \texttt{SELECT * FROM customers WHERE name LIKE "B\%";}
& \texttt{SELECT * FROM customers WHERE type IN (SELECT name FROM favorites);}
\newline
& \texttt{SELECT * FROM customers ORDER BY age DESC;}
& \texttt{SELECT * FROM customers WHERE weight < 80.0 ORDER BY age;}
\newline
& \texttt{SELECT name,\newline
>~~~CASE\newline
>~~~~~~~WHEN age > 65 THEN "OAP"\newline
>~~~~~~~WHEN age > 18 THEN "adult"\newline
>~~~~~~~ELSE "minor"\newline
>~~~END AS "ageType"\newline
>~~~FROM customers;}
\end{easylist}

\newpage
Aggregating Data:
\begin{easylist}[itemize]
\ListProperties(Style*=$\bullet$ , FinalMark={)})
& \texttt{SELECT SUM(weight) AS "totalWeight" FROM customers;}
& \texttt{SELECT \textbf{name}, COUNT(*) FROM students GROUP BY \textbf{name};}
\newline
& \texttt{SELECT \textbf{name}, MAX(age) AS "maxAge"\newline
>~~~FROM customers \textit{WHERE weight > 40.0}\newline
>~~~GROUP BY \textbf{name}\newline
>~~~HAVING maxAge > 50;}
% >~~~\textit{ORDER BY maxAge;}\newline
\newline
& \texttt{SELECT name,\newline
>~~~~~~weight/(SELECT SUM(weight) FROM customers) AS "fracWeight"\newline
>~~~FROM customers;}
\end{easylist}

\vspace{\baselineskip}
\vspace{\baselineskip}
\vspace{\baselineskip}
\vspace{\baselineskip}
Inner Join:\newline
\textit{Returns records that have matching values in both tables.\newline
Table order does not matter.}
\begin{easylist}[itemize]
\ListProperties(Style*=$\bullet$ , FinalMark={)})
& \texttt{SELECT * FROM students\newline
>~~~JOIN student\char`_grades\newline
>~~~ON students.id = student\char`_grades.student\char`_id;}
\end{easylist}

\vspace{\baselineskip}
Outer Join:\newline
\textit{Returns all records from the left table (first listed), and the matched records from the right table.\newline
Table order DOES matter.}
\begin{easylist}[itemize]
\ListProperties(Style*=$\bullet$ , FinalMark={)})
& \texttt{SELECT students.first\char`_name, students.last\char`_name, student\char`_projects.title\newline
>~~~FROM students\newline
>~~~LEFT OUTER JOIN student\char`_projects\newline
>~~~ON students.id = student\char`_projects.student\char`_id;}
\end{easylist}

\vspace{\baselineskip}
Self Join:\newline
\textit{Use an alias to perform self joins.}
\begin{easylist}[itemize]
\ListProperties(Style*=$\bullet$ , FinalMark={)})
& \texttt{SELECT students.name, \textbf{buddies}.email AS "buddy\char`_email"\newline
>~~~FROM students\newline
>~~~JOIN students \textbf{buddies}\newline
>~~~ON students.buddy\char`_id = \textbf{buddies}.id;}
\end{easylist}

\newpage
Updating Data:
\begin{easylist}[itemize]
\ListProperties(Style*=$\bullet$ , FinalMark={)})
& \texttt{UPDATE customers SET age = 33 WHERE id = 73;}
\end{easylist}

\vspace{\baselineskip}
Deleting Data:
\begin{easylist}[itemize]
\ListProperties(Style*=$\bullet$ , FinalMark={)})
& \texttt{DELETE FROM customers WHERE id = 73;}
\end{easylist}

\vspace{\baselineskip}
Add Column to Table:
\begin{easylist}[itemize]
\ListProperties(Style*=$\bullet$ , FinalMark={)})
& \texttt{ALTER TABLE customers ADD nickname TEXT;}
\end{easylist}

\vspace{\baselineskip}
The WITH clause is an optional prefix for SELECT.
You can then use the results in the main SELECT statement:
\begin{easylist}[itemize]
\ListProperties(Style*=$\bullet$ , FinalMark={)})
& \texttt{WITH query\char`_name (column\char`_name1, ...) AS\newline
>~~~(SELECT ...)\newline
SELECT ...}
\end{easylist}

\vspace{\baselineskip}
\underline{Window Functions:}\newline
They perform calculations over a set of rows.
They are indicated by the OVER clause.\newline
\newline
Here is what it all means:\newline
OVER = using all rows\newline
PARTITION BY = with the same...\newline
ORDER BY = applying function sequentially ordered by...
% 
\begin{easylist}[itemize]
\ListProperties(Style*=$\bullet$ , FinalMark={)})
& \texttt{SELECT firstname, lastname, date\char`_started,\newline
>~~~COUNT(*) OVER (PARTITION BY year(date\char`_started)) AS NumPerYear\newline
>~~~FROM Employee;}
\end{easylist}

\newpage